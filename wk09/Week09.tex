\documentclass[11pt,a4paper]{report}
\usepackage[utf8]{inputenc}
\usepackage[left=2cm,right=2cm,top=2cm,bottom=2cm]{geometry}
\usepackage{color}
\definecolor{mygreen}{rgb}{0,0.6,0}
\definecolor{mygray}{rgb}{0.5,0.5,0.5}
\definecolor{mymauve}{rgb}{0.58,0,0.82}

\usepackage{amsthm}
\theoremstyle{definition}
\newtheorem{exinn}{Example}[section]
\newenvironment{example}
{\clubpenalty=10000
	\begin{exinn}%
		\mbox{}%
		{\color{blue}\leaders\hrule height .8ex depth \dimexpr-.8ex+0.8pt\relax\hfill}%
		\mbox{}\linebreak\ignorespaces}
	{\par\kern2ex\hrule\end{exinn}}

\usepackage[english]{babel}
\usepackage{amsmath}
\usepackage{amsfonts}
\usepackage{amssymb}
\usepackage{mathtools}
\usepackage{tocloft}
\usepackage{listings}
\usepackage{graphicx}
\usepackage{tikz}
\usepackage{bigints}
\usepackage{fourier}
\usepackage{fancyhdr}
\pagestyle{fancy}
\usepackage{dsfont}
\usepackage{units}
\usepackage{textcomp}
\usepackage{subcaption}
\usepackage{parskip}
\usepackage{float}
\usepackage{pdfpages}
\renewcommand{\lstlistlistingname}{Code Listings}
\renewcommand{\lstlistingname}{Code Listing}
\definecolor{gray}{gray}{0.5}
\definecolor{green}{rgb}{0,0.5,0}
\lstset{
	tabsize=4,
	rulecolor=,
	language=python,
	%basicstyle=\ttfamily\scriptsize,
	basicstyle=\footnotesize,
	upquote=true,
	numbers=left,
	numberstyle=\footnotesize,
	aboveskip={1.5\baselineskip},
	extendedchars=true,
	linewidth=\linewidth,
	breaklines=false,
	prebreak=\raisebox{0ex}[0ex][0ex]{\ensuremath{\hookleftarrow}},
	frame=single,
	columns=fullflexible,
	showtabs=false,
	showspaces=false,
	showstringspaces=false,
	identifierstyle=\ttfamily,
	keywordstyle=\color[rgb]{0,0,1},
	commentstyle=\color[rgb]{0.133,0.545,0.133},
	stringstyle=\color[rgb]{0.627,0.126,0.941},
}
\pagestyle{fancy}
\lhead{Travis Mitchell}
\rhead{Week 09}
\chead{MECH3750 - Content Summary}
\renewcommand{\headrulewidth}{0.8pt}
\renewcommand{\footrulewidth}{0.8pt}

\author{\textit{Travis Mitchell}}
\title{Lecture Content Summaries for MECH3750}
\date{Updated: 20 September, 2019}			

\makeatletter
\newcommand*{\toccontents}{\@starttoc{toc}}
\makeatother
\renewcommand{\thesection}{\thepart \arabic{section}}


\begin{document}
\section{Hyperbolic PDEs}
\subsection{First order hyperbolic PDEs}
Here we will start by studying the \textbf{convection equation} which has many practical uses e.g. flows of air and water, transport phenomena and traffic behaviour. THe general form of the first order wave equation can be written as,
\begin{align*}
a\partial_t u + v \partial_x u = w.
\end{align*} 
In this equation we have, $u$ which is a multivariate function (e.g. a concentration), $v=v(x,t)$ is a velocity field and $w$ which is a source term. The coefficient $a$ is typically set to one. \\

In terms of the convection equation, it is often non-homogenous (i.e. $w\neq 0$) but can be homogenous,
\begin{align*}
\partial_t u + v \partial_x u = 0.
\end{align*}
Furthermore, the equation can be linear (if $v$ and $w$ are independent of $u$) or non-linear. A few examples of non-linearity include,
\begin{align*}
\partial t u + u \partial_x u = x \quad \quad \partial t u - x \partial_x u = 1/u.
\end{align*}

\subsection{Second order hyperbolic PDEs}
To study hyperbolic PDEs of second order, we will focus on the wave equation,
\begin{align*}
	\partial_{tt} u = c^2 \partial_{xx} u,
\end{align*}
where $c$ is the wave speed.
\subsubsection{Examples of where this is used:}
\textbf{Gas Dynamics} \\
If we start with our compressible Navier-Stokes equations, we can linearise these and introduce an equation of state to obtain,
\begin{align*}
	\partial_{tt} \rho = a^2 \partial_{xx} \rho, \quad a^2 = \left[\frac{dp}{d\rho}\right]_{s=s_0}
\end{align*}

\textbf{Oscillations (e.g. guitar string)} \\
In lectures we created a free body diagram of a \textit{discrete} representation of a continuous string. Applying Newton's second law of motion to these we are able to obtain,
\begin{align*}
	\partial_{tt} y = a^2 \partial_{xx} y, \quad a^2 = \left[\frac{F}{\rho A}\right]
\end{align*}

This is all well and good, but how do we actually solve these problems?

\subsection{Finite difference solution for the wave equation}
So starting with the wave equation, we can apply second order central differences to both the temporal and spatial derivatives,
\begin{align*}
\partial_{tt} u &= c^2 \partial_{xx} u \\
\frac{u_{i}^{m+1} - 2u_{i}^{m} + u_{i}^{m+1}}{\Delta t^2} &= c^2 \frac{u_{i}^{m+1} - 2u_{i}^{m} + u_{i}^{m+1}}{\Delta x^2}
\end{align*}

From here, we can arrange an explicit update for time, $m+1$. This involves multiplying by $\Delta t^2$ and taking terms to the right hand side giving,
\begin{align*}
	u_{i}^{m+1} &= 2u_{i}^{m} - u_{i}^{m-1} + c^2 \frac{\Delta t^2 (u_{i+1}^{m} - 2u_{i}^{m} + u_{i-1}^{m})}{\Delta t^2} \\
	u_{i}^{m+1} &= 2u_{i}^{m} - u_{i}^{m-1} + \frac{ c^2\Delta t^2 }{\Delta t^2} \left[u_{i+1}^{m} - 2u_{i}^{m} + u_{i-1}^{m}\right] \\
	u_{i}^{m+1} &= 2u_{i}^{m} - u_{i}^{m-1} + \sigma^2 \left[u_{i+1}^{m} - 2u_{i}^{m} + u_{i-1}^{m}\right] \\
	u_{i}^{m+1} &=\sigma^2 u_{i+1}^{m} + (2-2\sigma^2)u_{i}^{m} + \sigma^2u_{i-1}^{m}- u_{i}^{m-1} 
\end{align*}

It can be seen, that our general update depends not only on the previous time step, but the time step before this one as well. As such, we typically need both an initial position and velocity to solve this equation. The velocity, we approximate with a central difference and use to eliminate the term $u_i^{m-1}$. After this is done, a matrix equation can be formed as per usual.

\subsection{Non-reflecting boundary conditions}
When applying normal boundary conditions to the wave equation, it is possible that wave reflections back into the domain occur. Depending on the problem, this may not be physical and as such, schemes are required to let the waves propagate out of the domain. To understand the cause of (and ultimately how to eliminate) this issue, we consider the \textit{characteristics} of the wave equation. Therefore, we can analyse the wave equation as,
\begin{align*}
0 = \left(\partial_{tt} - c^2 \partial_{xx}\right) u = (\partial_t - c \partial_x)(\partial_t + c \partial_x)u
\end{align*}
Here we see that we actually have two \textit{one}-way convention equations that propagate with velocity, $c$ and $-c$. When analysing PDEs analytically, this so called \textit{Method-of-Characteristics} can be applied to determine a solution. Numerically though, understanding these allows us to remove them at the boundary resulting in a non-reflecting BC.

Therefore, we must solve for our boundary node using the outgoing convective equation and ignore the incoming wave. Thus on the left hand boundary condition,
\begin{align*}
	\partial_t u - c \partial_x u &= 0 \\
	\frac{u_0^{m+1} - u_0^m}{\Delta t} - c 	\frac{u_1^{m} - u_0^m}{\Delta x} &= 0 \\
	u_0^{m+1} &= u_o^m + \sigma (u_1^m - u_0^m)
\end{align*}
The same can be done for the right hand side.

\end{document}
