\documentclass[11pt,a4paper]{report}
\usepackage[utf8]{inputenc}
\usepackage[left=2cm,right=2cm,top=2cm,bottom=2cm]{geometry}
\usepackage{color}
\definecolor{mygreen}{rgb}{0,0.6,0}
\definecolor{mygray}{rgb}{0.5,0.5,0.5}
\definecolor{mymauve}{rgb}{0.58,0,0.82}

\usepackage{amsthm}
\theoremstyle{definition}
\newtheorem{exinn}{Example}[section]
\newenvironment{example}
{\clubpenalty=10000
	\begin{exinn}%
		\mbox{}%
		{\color{blue}\leaders\hrule height .8ex depth \dimexpr-.8ex+0.8pt\relax\hfill}%
		\mbox{}\linebreak\ignorespaces}
	{\par\kern2ex\hrule\end{exinn}}

\usepackage[english]{babel}
\usepackage{amsmath}
\usepackage{amsfonts}
\usepackage{amssymb}
\usepackage{mathtools}
\usepackage{tocloft}
\usepackage{listings}
\usepackage{graphicx}
\usepackage{tikz}
\usepackage{bigints}
\usepackage{fourier}
\usepackage{fancyhdr}
\pagestyle{fancy}
\usepackage{dsfont}
\usepackage{units}
\usepackage{textcomp}
\usepackage{subcaption}
\usepackage{parskip}
\usepackage{float}
\usepackage{pdfpages}
\renewcommand{\lstlistlistingname}{Code Listings}
\renewcommand{\lstlistingname}{Code Listing}
\definecolor{gray}{gray}{0.5}
\definecolor{green}{rgb}{0,0.5,0}
\lstset{
	tabsize=4,
	rulecolor=,
	language=python,
	%basicstyle=\ttfamily\scriptsize,
	basicstyle=\footnotesize,
	upquote=true,
	numbers=left,
	numberstyle=\footnotesize,
	aboveskip={1.5\baselineskip},
	extendedchars=true,
	linewidth=\linewidth,
	breaklines=false,
	prebreak=\raisebox{0ex}[0ex][0ex]{\ensuremath{\hookleftarrow}},
	frame=single,
	columns=fullflexible,
	showtabs=false,
	showspaces=false,
	showstringspaces=false,
	identifierstyle=\ttfamily,
	keywordstyle=\color[rgb]{0,0,1},
	commentstyle=\color[rgb]{0.133,0.545,0.133},
	stringstyle=\color[rgb]{0.627,0.126,0.941},
}
\pagestyle{fancy}
\lhead{Travis Mitchell}
\rhead{Week 04}
\chead{MECH3750 - Content Summary}
\renewcommand{\headrulewidth}{0.8pt}
\renewcommand{\footrulewidth}{0.8pt}

\author{\textit{Travis Mitchell}}
\title{Lecture Content Summaries for MECH3750}
\date{Updated: 15 August, 2019}			

\makeatletter
\newcommand*{\toccontents}{\@starttoc{toc}}
\makeatother
\renewcommand{\thesection}{\thepart \arabic{section}}


\begin{document}
\section{Fourier Series continued}
	\begin{example}
		\textit{Find the Fourier Series of} $f(x) = x^2$ on $[-\pi, \pi]$. \\\\
		This example shows a key point to note (particularly for quizzes/exams), and that is to check if the function is \textit{odd} or \textit{even}! The first step here is to determine our Fourier coefficients:
		\begin{align*}
			b_j &= \frac{1}{\pi} \int_{-\pi}^{\pi} \underbrace{x^2}_{even} \underbrace{\sin(j x)_{odd} dx} \\
				&= 0 \\
			a_j &= \frac{1}{\pi} \int_{-\pi}^{\pi} \underbrace{x^2}_{even} \underbrace{\cos(j x)_{even} dx} \\
			&= \frac{2}{\pi} \int_{0}^{\pi} x^2 \cos(j x) dx 
		\end{align*}
		The coefficients, $a_j$, can then be determined through integration by parts to be,
		\begin{align*}
			a_j = \left( \frac{4(-1)^{j}}{j^2} \right)
		\end{align*}
		We now evaluate, $a_0$, and then we can determine the Fourier series,
		\begin{align*}
			a_0&= \frac{2}{\pi} \int_0^{\pi} x^2 dx\\
			   &= 2\pi^2/3
		\end{align*}
		And therefore, as $f(x)$ is even,
		\begin{align*}
			f(x)&= a_0/2 + \sum_{n=1} ^{\infty} a_n cos(nx)\\
				&= \pi^2/3 - \frac{4\cos(x)}{1^2} + \frac{4\cos(2x)}{2^2} - \frac{4\cos(3x)}{3^2} + \cdots
		\end{align*}
	\end{example}

	\textbf{Remarks from Fourier Theorem,} namely is we can restate the formulation as,
	\begin{align*}
		f(x) &= \sum_{j=-n}^{n} c_j e^{ijx} \\
			 &c_j = \frac{1}{2\pi} \int_{-\pi}^{\pi} f(x) e^{-ijx} dx, \quad j=-n,\dots, -1, 0,1,\dots,n
	\end{align*}
	This makes use of Euler's formula (or we can also show with Taylor series) that, $e^{ijx} = \cos{jx} + i \sin(jx)$. \\
	
	\subsection{Extension to arbitrary domain}
	Here we first state the result for the interval $[-L,L]$,
	\begin{align*}
		f(x) &= \frac{a_0}{2} + \sum_{j=1}^{\infty} a_j \cos \left(\frac{j\pi x}{L}\right) + b_j \sin \left(\frac{j\pi x}{L}\right),\\
			&a_j = \frac{1}{L} \int_{-L}^{L} \cos \left(\frac{j\pi x}{L}\right) f(x) dx, \quad j=0,1,2,\dots \\
			&b_j = \frac{1}{L} \int_{-L}^{L} \sin \left(\frac{j\pi x}{L}\right) f(x) dx
	\end{align*}
	To come to this result, we simply make a transformation in which we search for the Fourier series of $F(z)$ on the domain $[-\pi,\pi]$, but set $f(x)=F(z)$ with $z=\pi x/L$.
\section{Discrete Fourier Transform}
\end{document}
