\documentclass[11pt,a4paper]{report}
\usepackage[utf8]{inputenc}
\usepackage[left=2cm,right=2cm,top=2cm,bottom=2cm]{geometry}
\usepackage{color}
\definecolor{mygreen}{rgb}{0,0.6,0}
\definecolor{mygray}{rgb}{0.5,0.5,0.5}
\definecolor{mymauve}{rgb}{0.58,0,0.82}

\usepackage{amsthm}
\theoremstyle{definition}
\newtheorem{exinn}{Example}[section]
\newenvironment{example}
{\clubpenalty=10000
	\begin{exinn}%
		\mbox{}%
		{\color{blue}\leaders\hrule height .8ex depth \dimexpr-.8ex+0.8pt\relax\hfill}%
		\mbox{}\linebreak\ignorespaces}
	{\par\kern2ex\hrule\end{exinn}}

\usepackage[english]{babel}
\usepackage{amsmath}
\usepackage{amsfonts}
\usepackage{amssymb}
\usepackage{mathtools}
\usepackage{tocloft}
\usepackage{listings}
\usepackage{graphicx}
\usepackage{tikz}
\usepackage{bigints}
\usepackage{fourier}
\usepackage{fancyhdr}
\pagestyle{fancy}
\usepackage{dsfont}
\usepackage{units}
\usepackage{textcomp}
\usepackage{subcaption}
\usepackage{parskip}
\usepackage{float}
\usepackage{pdfpages}
\renewcommand{\lstlistlistingname}{Code Listings}
\renewcommand{\lstlistingname}{Code Listing}
\definecolor{gray}{gray}{0.5}
\definecolor{green}{rgb}{0,0.5,0}
\lstset{
	tabsize=4,
	rulecolor=,
	language=python,
	%basicstyle=\ttfamily\scriptsize,
	basicstyle=\footnotesize,
	upquote=true,
	numbers=left,
	numberstyle=\footnotesize,
	aboveskip={1.5\baselineskip},
	extendedchars=true,
	linewidth=\linewidth,
	breaklines=false,
	prebreak=\raisebox{0ex}[0ex][0ex]{\ensuremath{\hookleftarrow}},
	frame=single,
	columns=fullflexible,
	showtabs=false,
	showspaces=false,
	showstringspaces=false,
	identifierstyle=\ttfamily,
	keywordstyle=\color[rgb]{0,0,1},
	commentstyle=\color[rgb]{0.133,0.545,0.133},
	stringstyle=\color[rgb]{0.627,0.126,0.941},
}
\pagestyle{fancy}
\lhead{Travis Mitchell}
\rhead{Week 06}
\chead{MECH3750 - Content Summary}
\renewcommand{\headrulewidth}{0.8pt}
\renewcommand{\footrulewidth}{0.8pt}

\author{\textit{Travis Mitchell}}
\title{Lecture Content Summaries for MECH3750}
\date{Updated: 22 August, 2019}			

\makeatletter
\newcommand*{\toccontents}{\@starttoc{toc}}
\makeatother
\renewcommand{\thesection}{\thepart \arabic{section}}


\begin{document}
\section{Separation of Variables Conti.}
Last week we managed to conclude:
\begin{align*}
 u(x,t) = F(x)G(t) &= b\sin(n\pi x/L) [A \cos(n\pi c t/L) + B \sin(n\pi ct/L)] \\
 				   &= [A_n \cos(n\pi c t/L) + B_n \sin(n\pi ct/L)] \sin(n\pi x/L)
\end{align*}

\textit{Aside:} A note on linearity \\
As each value of $n$ gives us a new solution to the equation, as they are linear equations, the superposition of these is also a solution. This means we can write,
\begin{align*}
	u(x,t) = \sum_1^{\infty} \left(A_n \cos \left(\frac{n\pi c t}{L}\right) + B_n \sin \left(\frac{n\pi c t}{L}\right)\right) \sin \left(\frac{n\pi x}{L}\right).
\end{align*}

Here, we now apply initial conditions to get our coefficients. This aligns quite closely with our past work on Fourier series. We obtain:
\begin{align*}
	A_n= \frac{2}{L} \int_0^L \sin \left(\frac{n\pi x}{L}\right) f(x) dx
\end{align*}
where, $f(x)= u(x,0)$. The other initial condition, $g(x) = u_t(x,0)$ gives,
\begin{align*}
	B_n = \frac{2}{n\pi c} \int_0^L \sin \left(\frac{n\pi x}{L}\right) g(x) dx
\end{align*} 


\section{Fourier's method for 1-D heat equation}
Looking to solve:
\begin{align*}
	\partial_t u = k \partial_{xx} u
\end{align*}
and we are considering thermally insulated ends (boundary conditions),
\begin{align*}
	\partial_x u(0,t) &= 0, \\
	\partial_x u(L,t) &= 0.
\end{align*}
We also have the initial condition,
\begin{align*}
	u(x,0) = f(x).
\end{align*}

As per the previous case, we are assuming we can find a solution in a separable form:
\begin{align*}
	u(x,t) = F(x) G(t).
\end{align*}
Now apply the boundary conditions again and eliminate useless solutions to obtain,
\begin{align*}
	 F'(0) = F'(L) = 0.
\end{align*}

Now formulate the partials and sub them into the heat equation. Separating our variables onto the left and right hand side again gives us,
\begin{align*}
	\frac{G'}{c^2G} = \frac{F''}{F}.
\end{align*}
This is again a function of $t$ equal to a function of $x$, so it is only possible if they equal a constant. Therefore, we now have two partial differential equations that we know how to solve,
\begin{align*}
	G' &= kc^2 G \\
	F''&= kF, \quad F'(0)=F'(L)=0.
\end{align*}

If we go through the process, we again find that the constant needs to be negative so for convenience set, $k=-p^2$. Thus we need to solve,
\begin{align*}
	F'' &= -p^2 F \\
	\therefore F &= a \cos(px) + b\sin(px).
\end{align*}
Applying our boundary conditions for $F$ gives,
\begin{align*}
	b&=0, \quad p = \frac{n\pi}{L}, \\
\implies	F(x) &= a\cos\left(\frac{n\pi x}{L}\right).
\end{align*}
Therefore, our temporal function becomes
\begin{align*}
	G' &= kc^2 G \\
	\implies G(t) &= D \exp\left[-\left(\frac{n\pi c}{L}\right)^2 t\right]
\end{align*}
Combining these equations we get,
\begin{align*}
u(x,t) &= A_0 \\
u_n(x,t) &= A_n\exp\left[-\left(\frac{n\pi c}{L}\right)^2 t\right]\cos\left(\frac{n\pi x}{L}\right). \\
\therefore u(x,t) &= A_0 + \sum_n u_n.
\end{align*}
This currently satisfies the PDE and the BCs, so we then use the initial condition to define the coefficients. From this,
\begin{align*}
	A_n = \frac{2}{L} \int_0^L f(x) \cos \frac{n\pi x}{L} dx
\end{align*}

\subsection{Steady-state, 2D heat equation}
We can apply the same ideas to,
\begin{align*}
	u_{xx} + u_{yy} = 0.
\end{align*}

Doing the normal procedure, $u(x,y) = F(x)G(y)$ we achieve the two ODEs,
\begin{align*}
	F'' &= kF \\
	G'' &= -kG.
\end{align*}
Again, taking $k$ as negative, we find,
\begin{align*}
	F &= A\cos(px) + B\sin(px), \quad F(0) = F(a) = 0 \\
	\implies F(x) &= B_n \sin\left(\frac{n\pi}{a}x\right) \\
	G &= C \cosh\left(\frac{n\pi y}{a}\right) + D\sinh\left(\frac{n\pi y}{a}\right), \quad G(0) = 0 \\
	G(y) &= D\sinh\left(\frac{n\pi y}{a}\right).
\end{align*}
So combining these,
\begin{align*}
	u(x,y) = \sum_n B_n \sin\left(\frac{n\pi}{a}x\right)\sinh\left(\frac{n\pi }{a}y\right)
\end{align*}

After this process there is conveniently one BC left to fix $B_n$ at $u(x,b)=f(x)$. Using a Fourier sine series we find,
\begin{align*}
	B_n = \frac{2}{a\sinh(2\pi b/a)} \int_0^a f(x) \sin\frac{n\pi x}{a} dx
\end{align*}

\end{document}
